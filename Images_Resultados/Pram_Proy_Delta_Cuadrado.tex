
\begin{figure}[h!]
     \centering
    \captionsetup[sub]{font=small}
     \begin{minipage}[h!]{0.7\textwidth}
         \begin{subfigure}[b!]{0.44 \textwidth}
             \caption{$\epsilon = 1\%$}
             \includegraphics[width=\textwidth]{Imagenes/Resultados_Hoti_Cuadrado/bands_square_shh_0.01.pdf}
             \label{}
         \end{subfigure}\hspace*{-0.5em}
         \begin{subfigure}[b!]{0.56 \textwidth}
             \caption*{}
             \includegraphics[width=\textwidth]{Imagenes/Resultados_Hoti_Cuadrado/proyection_square_0.01.pdf}
             \label{}
         \end{subfigure}\hspace*{-0.5em}
     \end{minipage}\vspace*{-2.5em}
     
     \begin{minipage}[h!]{0.7\textwidth}
         \begin{subfigure}[b!]{0.44 \textwidth}
             \caption{$\epsilon = 5\%$}
             \includegraphics[width=\textwidth]{Imagenes/Resultados_Hoti_Cuadrado/bands_square_shh_0.05.pdf}
             \label{}
         \end{subfigure}\hspace*{-0.5em}
         \begin{subfigure}[b!]{0.56 \textwidth}
             \caption*{}
             \includegraphics[width=\textwidth]{Imagenes/Resultados_Hoti_Cuadrado/proyection_square_0.05.pdf}
             \label{}
         \end{subfigure}\hspace*{-0.5em}
     \end{minipage}\vspace*{-2.5em}
     
     \begin{minipage}[h!]{0.7\textwidth}
         \begin{subfigure}[b!]{0.44 \textwidth}
             \caption{$\epsilon = 10\%$}
             \includegraphics[width=\textwidth]{Imagenes/Resultados_Hoti_Cuadrado/bands_square_shh_0.1.pdf}
             \label{}
         \end{subfigure}\hspace*{-0.5em}
         \begin{subfigure}[b!]{0.56 \textwidth}
             \caption*{}
             \includegraphics[width=\textwidth]{Imagenes/Resultados_Hoti_Cuadrado/proyection_square_0.1.pdf}
             \label{}
         \end{subfigure}\hspace*{-0.5em}
     \end{minipage}\vspace*{-2.5em}
     
     \begin{minipage}[h!]{0.7\textwidth}
         \begin{subfigure}[b!]{0.44 \textwidth}
             \caption{$\epsilon = 30\%$}
             \includegraphics[width=\textwidth]{Imagenes/Resultados_Hoti_Cuadrado/bands_square_shh_0.3.pdf}
             \label{}
         \end{subfigure}\hspace*{-0.5em}
         \begin{subfigure}[b!]{0.56 \textwidth}
             \caption*{}
             \includegraphics[width=\textwidth]{Imagenes/Resultados_Hoti_Cuadrado/proyection_square_0.3.pdf}
             \label{}
         \end{subfigure}\hspace*{-0.5em}
     \end{minipage}\vspace*{-2.5em}
     
     \begin{minipage}[h!]{0.7\textwidth}
         \begin{subfigure}[b!]{0.44 \textwidth}
             \caption{$\epsilon = 50\%$}
             \includegraphics[width=\textwidth]{Imagenes/Resultados_Hoti_Cuadrado/bands_square_shh_0.5.pdf}
             \label{}
         \end{subfigure}\hspace*{-0.5em}
         \begin{subfigure}[b!]{0.56 \textwidth}
             \caption*{}
             \includegraphics[width=\textwidth]{Imagenes/Resultados_Hoti_Cuadrado/proyection_square_0.5.pdf}
             \label{}
         \end{subfigure}\hspace*{-0.5em}
     \end{minipage}\vspace*{-2em}
     
     
       \caption{En la columna derecha se muestra el espectro de energías en una red cuadrada como función de $\gamma/\lambda $ con un desorden aleatorio en los parámetros de salto, de tamaño $\delta$. Las lineas rojas corresponden a los cuatro estados degenerados que representan los estados localizados en las esquinas. En la columna izquierda se muestra la densidad de probabilidad de la fase no trivial donde $\gamma' = \gamma( 1 \pm \epsilon) ,\, \, \lambda' = \lambda( 1 \pm \epsilon)$.  }
    \label{fig:Pram_Proy_Delta_cuadrado}
\end{figure}