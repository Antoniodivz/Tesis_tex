\chapter{Conclusiones}  %

%\textit{Aqui pondras tus concluciones, como se comporto el multipolo fractal, etc.} 

Es posible repoducir los estados topologicos de orden superior, que se presentan en una red cristalina cuadrada,  por medio del modelo de BBH en una red fractal de Sierpinski, estos estados son robustos frente a perturbaciones en los parametros de salto, que se robustecen aún más conforme el tamaño de la red aumenta con las genraciones del fractal, lo cual tambien hace que los estados de los borde interiores (borde III) del fractal comiencen a robustecerse debido a la significacia del borde con respecto al tamaño de la red. Estos estados de bode se ven reflejados en densidades concentradas en las esquinas exteriores de la red con probabilidad de aproximadamente $1/4$ alcanzando su maximo ($\rho_{corner} = 1/4 $) en el caso donde las celdas estan totalmente cuatrimerizadas, es decir, cuando $\gamma/\lambda = 0$.

También fue posible estudiar el bombeo de las densidades de probabilidad, a través del modelo de BBH con parametros de salto y de sitio ciclicos, esto como el resultado de las transiciones entre fase topológica a trivial y de trivial a topológica conforme cambia el parametro ciclico $\theta$. Las propiedades de bombeo se presentaron tanto en la red cristalina cuadrada como en la red fractal de Sierpinski, en espacio fase del flujo de la densidad se pudo determinar que los parametros para los cuales el bombeo era beneficiado son cuando la transicion sucede en los puntos cercanos a $\theta =  \pi/2$ siendo $\gamma/\lambda \approx 1$ en el caso cuadrado y $\gamma/\lambda \approx 0.3$ en el caso fractal. Este bombeo se ve reflejado como el desplazamiento del valor esperado de la posiciones o centros de wannier (cuando tenemos simetría traslacional) a través de las celdas que componen las redes. Este desplazamiento se obtiene cuando el bombeo se realiza sobre cualquier dirección (\textbf{XY}, \textbf{X} o \textbf{Y}) y se realiza en la direccion misma de la variacion de los parametros, este comportamiento es muy claro en la red cauadrada, sin embargo en la red fractal no es muy claro cuando la variación de los parametros es unidireccional. Sin embargo podemos concluir que la mayoria de caractaristicas del aislante topologico de orden superior en la red cirstalina fractal se presentan en la red Fractal se Sierpinski que no tiene siemetria traslacional y tiene más bordes que crecen confrorme se tiene mas generaciones, estas caracteristicas aparecen como consecuencia de que las caractaristicas topologicas de las celdas unitarias que confroman las distintas redes son las mismas. 


\clearpage % or \cleardoublepage