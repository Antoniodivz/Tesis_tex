\chapter{Resultados}

\begin{center}
\begin{minipage}{0.9\textwidth}
{\small
{\bf Resumen:} Resumen de resultados.
}
\end{minipage}
\end{center}



\subsection{Hoti Cuadrado}
%{\it Primero mencionar qué se grafica en la figura antes interpretarla.} Como se puede observar en la figura \ref{fig:Dos_cuadrado} la solución del modelo de SSH para el arreglo cuadrado (figura) arroja una clara transición de fase electrónica conforme los parámetros de salto $\gamma/\lambda$ varían. Para valores de $\gamma/\lambda < 1$ podemos observar 4 estados en el centro lo cual indicaría una fase topológica estos estados carecen de brecha energética, estos estados se deben a finitud geométrica del arreglo. Para valores de $\gamma/\lambda = 0$ tendríamos el caso limite en el que tenemos un fase semimetálica y para valores de $\gamma/\lambda > 1$ tendríamos una fase aislante o fase trivial.
\begin{figure}[h!]
     \centering
    \captionsetup[sub]{font=small}
     \begin{subfigure}[b!]{0.3 \textwidth}
         \caption{}
         \includegraphics[width=\textwidth]{Imagenes/Resultados_Hoti_Cuadrado/bars_square1.pdf}
     \end{subfigure}\hspace*{1em}
     \begin{subfigure}[b!]{0.3 \textwidth}
         \caption{}
         \includegraphics[width=\textwidth]{Imagenes/Resultados_Hoti_Cuadrado/bars_square2.pdf}
     \end{subfigure}\hspace*{1em}
     \begin{subfigure}[b!]{0.3 \textwidth}
         \caption{}
         \includegraphics[width=\textwidth]{Imagenes/Resultados_Hoti_Cuadrado/bars_square3.pdf}
     \end{subfigure}\hspace*{1em}\vspace*{-0.5em}
        \caption{Numero de estados por energía en un red con geometría cuadrada para diferentes valores de los parámetros de salto, \textbf{a)} $\gamma /\lambda = 1/2$, \textbf{b)} $\gamma /\lambda = 1$, \textbf{c)} $\gamma /\lambda = 2$.}
        \label{fig:Dos_Cuadrado}
\end{figure}
En la figura \ref{fig:Dos_Cuadrado} podemos ver el numero de estados por energía en el modelo de BBT para una red cuadrada \ref{Modelo_SSH_squara_and_Fractal} para diferentes valores de $\gamma/\lambda$. Para los valores de $\gamma/\lambda>1$ \ref{fig:Dos_Cuadrado} \texbf{c)} podemos ver un claro gap energético, lo cual indicaría que nuestra red cristalina se encuentra en un estado trivial, es decir, es un aislante común. Para los valores de $\gamma/\lambda<1$ \ref{fig:Dos_Cuadrado} \texbf{a)} aparecen 4 estados degenerados carentes de gap en la energía de Fermi, estos aparecen como consecuencia de la finitud de la red cristalina por lo que les llamaremos estados de borde, esta fase del la red es conocida como fase topológica. En la figura \ref{fig:Dos_Cuadrado} para  \texbf{a)} $\gamma/\lambda = 1$ vemos que el gap energético de la red cristalina se cierra, lo cual indica la posible transición de fase entre el estado topológico y trivial con la variación de los parámetros de salto $\gamma/\lambda$ de la red cristalina.


\begin{figure}[h!]
     \centering
    \captionsetup[sub]{font=small}
     \begin{subfigure}[b!]{0.44 \textwidth}
         \caption{}
         \includegraphics[width=\textwidth]{Imagenes/Resultados_Hoti_Cuadrado/bands_square_shh.pdf}
     \end{subfigure}\hspace*{1em}
     \begin{subfigure}[b!]{0.56 \textwidth}
         \caption{}
         \includegraphics[width=\textwidth]{Imagenes/Resultados_Hoti_Cuadrado/proyection_square.pdf}
     \end{subfigure}\hspace*{1em} \vspace*{-0.5em}
        \caption{\textbf{(a)}Espectro de energía del sistema con geometría cuadrada y condiciones abiertas a la frontera, como función de $\gamma/\lambda$. Las energías de borde coloreadas en Azul-Rojo corresponde a los 4 estados localizados en las esquinas. \textbf{(b)} Densidad de probabilidad en un fase no trivial donde $\gamma = 0.5,\, \, \lambda = 1$, en un red cuadrada de $18\times18$ sitios.}
    \label{fig:Pram_Proy_cuadrado}
\end{figure}



Para ver mas a detalle como sucede esta transición de fase podemos hacer un seguimiento del espectro de energías conforme se varían los parámetros de salto y enfocarnos en las energías centrales correspondientes a los estados de borde Figura \ref{fig:Pram_Proy_cuadrado} \texbf{(a)}. Las energías centrales están coloreadas con azul en la zona trivial y rojo en la zona topológica, en la zona $|\gamma/\lambda|>1$ el comportamiento de estas energías es similar a las energías de los estados del bulto de la red cristalna, sin embargo, en la zona $|\gamma/\lambda|>1$ las energías centrales se degeneran en la energía de Fermi, al proyectar las densidades de probabilidad de los estados correspondientes a estas 4 energías obtenemos la figura \ref{fig:Pram_Proy_cuadrado} \texbf{(b)} donde las densidades se encuentran claramente localizadas en las equinas de la red con probabilidad de aproximadamente $1/4$, lo cual definiria a nuestra red cirstalina como un aislante topológico de orden superior (HOTI \footnote{Siglas en ingles de High Order Topological Insulator}), donde el valor máximo de $\rho_{corner} = 1/4$ se alcanza en el caso limite $|\gamma/\lambda| = 0$ donde las celdas estarían totalmente cuatrimerizadas.

También estos estados son robustos ante perturbaciones como se puede observar en la figura \ref{fig:para_proy_Delta_fractal} \texbf{(a)-(e)} donde las perturbaciones se ejercieron sobre los parámetros de salto $\gamma' = \gamma( 1 \pm \epsilon) ,\, \, \lambda' = \lambda( 1 \pm \epsilon)$, \comA{Simetrias}, como se puede ver estas variaciones solo cambian las energías correspondientes a los estados del bulto y no a los estados de borde que permanecen robustos, así como en la proyección de la densidad de probabilidad en el espacio real se mantienen inalterados los estados localizados en las esquinas.



\begin{figure}[h!]
     \centering
    \captionsetup[sub]{font=small}
     \begin{minipage}[h!]{0.7\textwidth}
         \begin{subfigure}[b!]{0.44 \textwidth}
             \caption{$\epsilon = 1\%$}
             \includegraphics[width=\textwidth]{Imagenes/Resultados_Hoti_Cuadrado/bands_square_shh_0.01.pdf}
             \label{}
         \end{subfigure}\hspace*{-0.5em}
         \begin{subfigure}[b!]{0.56 \textwidth}
             \caption*{}
             \includegraphics[width=\textwidth]{Imagenes/Resultados_Hoti_Cuadrado/proyection_square_0.01.pdf}
             \label{}
         \end{subfigure}\hspace*{-0.5em}
     \end{minipage}\vspace*{-2.5em}
     
     \begin{minipage}[h!]{0.7\textwidth}
         \begin{subfigure}[b!]{0.44 \textwidth}
             \caption{$\epsilon = 5\%$}
             \includegraphics[width=\textwidth]{Imagenes/Resultados_Hoti_Cuadrado/bands_square_shh_0.05.pdf}
             \label{}
         \end{subfigure}\hspace*{-0.5em}
         \begin{subfigure}[b!]{0.56 \textwidth}
             \caption*{}
             \includegraphics[width=\textwidth]{Imagenes/Resultados_Hoti_Cuadrado/proyection_square_0.05.pdf}
             \label{}
         \end{subfigure}\hspace*{-0.5em}
     \end{minipage}\vspace*{-2.5em}
     
     \begin{minipage}[h!]{0.7\textwidth}
         \begin{subfigure}[b!]{0.44 \textwidth}
             \caption{$\epsilon = 10\%$}
             \includegraphics[width=\textwidth]{Imagenes/Resultados_Hoti_Cuadrado/bands_square_shh_0.1.pdf}
             \label{}
         \end{subfigure}\hspace*{-0.5em}
         \begin{subfigure}[b!]{0.56 \textwidth}
             \caption*{}
             \includegraphics[width=\textwidth]{Imagenes/Resultados_Hoti_Cuadrado/proyection_square_0.1.pdf}
             \label{}
         \end{subfigure}\hspace*{-0.5em}
     \end{minipage}\vspace*{-2.5em}
     
     \begin{minipage}[h!]{0.7\textwidth}
         \begin{subfigure}[b!]{0.44 \textwidth}
             \caption{$\epsilon = 30\%$}
             \includegraphics[width=\textwidth]{Imagenes/Resultados_Hoti_Cuadrado/bands_square_shh_0.3.pdf}
             \label{}
         \end{subfigure}\hspace*{-0.5em}
         \begin{subfigure}[b!]{0.56 \textwidth}
             \caption*{}
             \includegraphics[width=\textwidth]{Imagenes/Resultados_Hoti_Cuadrado/proyection_square_0.3.pdf}
             \label{}
         \end{subfigure}\hspace*{-0.5em}
     \end{minipage}\vspace*{-2.5em}
     
     \begin{minipage}[h!]{0.7\textwidth}
         \begin{subfigure}[b!]{0.44 \textwidth}
             \caption{$\epsilon = 50\%$}
             \includegraphics[width=\textwidth]{Imagenes/Resultados_Hoti_Cuadrado/bands_square_shh_0.5.pdf}
             \label{}
         \end{subfigure}\hspace*{-0.5em}
         \begin{subfigure}[b!]{0.56 \textwidth}
             \caption*{}
             \includegraphics[width=\textwidth]{Imagenes/Resultados_Hoti_Cuadrado/proyection_square_0.5.pdf}
             \label{}
         \end{subfigure}\hspace*{-0.5em}
     \end{minipage}\vspace*{-2em}
     
     
       \caption{En la columna derecha se muestra el espectro de energías en una red cuadrada como función de $\gamma/\lambda $ con un desorden aleatorio en los parámetros de salto, de tamaño $\delta$. Las lineas rojas corresponden a los cuatro estados degenerados que representan los estados localizados en las esquinas. En la columna izquierda se muestra la densidad de probabilidad de la fase no trivial donde $\gamma' = \gamma( 1 \pm \epsilon) ,\, \, \lambda' = \lambda( 1 \pm \epsilon)$.  }
    \label{fig:Pram_Proy_Delta_cuadrado}
\end{figure}



\subsection{Hoti Fractal}

\begin{figure}[h!]
     \centering
    \captionsetup[sub]{font=small}
     \begin{subfigure}[b!]{0.3 \textwidth}
         \caption{}
         \includegraphics[width=\textwidth]{Imagenes/Resultados_Hoti_Fractal/bars_square1.pdf}
         \label{}
     \end{subfigure}\hspace*{1em}
     \begin{subfigure}[b!]{0.3 \textwidth}
         \caption{}
         \includegraphics[width=\textwidth]{Imagenes/Resultados_Hoti_Fractal/bars_square2.pdf}
         \label{}
     \end{subfigure}\hspace*{1em}
     \begin{subfigure}[b!]{0.3 \textwidth}
         \caption{}
         \includegraphics[width=\textwidth]{Imagenes/Resultados_Hoti_Fractal/bars_square3.pdf}
         \label{}
     \end{subfigure}\hspace*{1em}\vspace*{-1.5em}
        \caption{Cantidad de estados por energía en un red con geometría Fractal (2da generación) para diferentes valores de los parámetros de salto, \textbf{a)} $\gamma /\lambda = 1/2$, \textbf{b)} $\gamma /\lambda = 1$, \textbf{c)} $\gamma /\lambda = 2$.}
    \label{fig:Dos_fractal}
\end{figure}


En la figura \ref{fig:Dos_fractal} observamos el numero de estados por energía para el modelo de la red de Sierpinski para distintos valores de los parámetros de salto. Para valores de $\gamma/\lambda < 1$ seguimos teniendo una fase topológica con 4 estados de borde carentes de gap sobre la energía de Fermi que aparecen en la red cuadrada, sin embargo aparecen mas estados al rededor a la energía de Fermi que corresponden a nuevos estados de borde que emergen como consecuencia a los huecos en el bulto de la construcción del fractal. También podemos ver que para valores $\gamma/\lambda > 1$, al igual que en la red cuadrada, tenemos un histograma correspondiente a una fase trivial, en medio de esta transición tenemos $\gamma/\lambda = 1$ donde se puede ver como el gap las energías se comienzan a cerrar, esto se puede ver mas a detalle en las figura \ref{fig:Param_Proy_fractal} \textbf{(a)} donde se puede apreciar el cambio del espectro de las energías conforme se varían los parámetros de salto, en esta figura podemos notar como a partir de estos valores de $\gamma/\lambda$ se comienza a distinguir el comportamiento de borde del comportamiento de bulto. Notemos que las energías correspondiente a los estados de borde se comienzan a cerrar hasta degenerarse en la energía de fermi cuando $\gamma/\lambda = 0$ fig. , es decir cuando las celdas están totalmente cuatrimerizadas \ref{fig:Param_Proy_fractal} \textbf{(b)}, para evitar este caso es necesario que la elección de los cuatro estados centrales correspondientes a la acumulación de la densidad de estados en las esquinas se sintonice en puntos de $\gamma/\lambda \in [0.5,0)$ fig. \ref{fig:Param_Proy_fractal} \textbf{(c)}. Conforme nos acerquemos al caso limite de $\gamma/\lambda = 0$ la densidad de de probabilidad acumulada en las esquinas $\rho_{corner} \approx 1/4$ 


\begin{figure}[h!]
     \centering
    \captionsetup[sub]{font=small}
     \begin{subfigure}[b!]{0.2 \textwidth}
         \caption{}
         \includegraphics[width=\textwidth]{Imagenes/Resultados_Hoti_Fractal/bands_square_shh.pdf}
     \end{subfigure}\hspace*{-0.5em}
     \begin{subfigure}[b!]{0.2 \textwidth}
         \caption{}
         \includegraphics[width=\textwidth]{Imagenes/Resultados_Hoti_Fractal/bands_square_shh_log.pdf}
     \end{subfigure}\hspace*{-0.5em} 
      \begin{subfigure}[b!]{0.3 \textwidth}
         \caption{}
         \includegraphics[width=\textwidth]{Imagenes/Resultados_Hoti_Fractal/proyection_square.pdf}
     \end{subfigure}\hspace*{-0.5em} 
     \begin{subfigure}[b!]{0.3 \textwidth}
        \caption{}
        \includegraphics[width=\textwidth]{Imagenes/Resultados_Hoti_Fractal/proyection_square_other.pdf}
    \end{subfigure}\hspace*{-0.5em} \vspace*{-0.5em}
        \caption{\textbf{(a)} Espectro de energía del sistema con geometría fractal y condiciones abiertas a la frontera, como función de $\gamma/\lambda$. \textbf{(b)} Espectro de energía en escala logarítmica del sistema antes mencionado. Las lineas rojas corresponden a los cuatro estados degenerados que representan los estados localizados en las esquinas. \textbf{(c)} Densidad de probabilidad en un fase no trivial donde $\gamma = 1,\, \, \lambda = 4.5$, en un red fractal de 2da generación correspondientes a los estados de borde I. \textbf{(d)} Densidad de probabilidad en un fase no trivial donde $\gamma = 1,\, \, \lambda = 4.5$, en un red fractal de 2da generación correspondientes a los estados de borde III.}
\label{fig:Param_Proy_fractal}
\end{figure}

Si pensamos a la red de Sierpinski como una red cuadrada que fue sometida a transformaciones espaciales discontinuas como lo es abrirle agujeros, podemos asegurar que es posible sintonizar los estados que permiten la realización del HOTI \comA{Momento cuadrupolar} para ciertos parámetros de $\gamma/\lambda$, es decir estas aun con perturbaciones de este tipo es posible conseguir que las densidades de estado se encuentren en localizadas en las esquinas.
Sin embargo estos estados también son robustos ante perturbaciones en los parámetros de salto, aunque en menor medida, como se puede ver en la figura \ref{fig:para_proy_Delta_fractal} \textbf{(a)-(e)} conforme $\epsilon$ crece las perturbaciones afectan primero a las energías del bulto y posteriormente deformando la trayectorias de las lineas correspondientes a las energías de borde II y III, sin embargo las energías de borde de la zona I correspondiente a los estados en las esquinas no es modificada, sin embargo al proyectar la densidades de probabilidad sobre la red cristalina podemos ver que para variaciones con $\epsilon \leq 30\%$ se generan asimetrías en las las acumulaciones de densidades que se encuentran distribuidas en las fronteras internas de la red.



\begin{figure}[h!]
     \centering
    \captionsetup[sub]{font=small}
     \begin{minipage}[h!]{0.9\textwidth}
         \begin{subfigure}[b!]{0.3 \textwidth}
            \caption{$\epsilon = 1\%$}             \includegraphics[width=\textwidth]{Imagenes/Resultados_Hoti_Fractal/bands_square_shh_0.05.pdf}
         \end{subfigure}\hspace*{-0.5em}
         \begin{subfigure}[b!]{0.3 \textwidth}
             \caption*{}
             \includegraphics[width=\textwidth]{Imagenes/Resultados_Hoti_Fractal/bands_square_shh_log0.05.pdf}
         \end{subfigure}\hspace*{-0.5em}
         \begin{subfigure}[b!]{0.4 \textwidth}
            \caption*{}
            \includegraphics[width=\textwidth]{Imagenes/Resultados_Hoti_Fractal/proyection_square_0.05.pdf}
         \end{subfigure}\hspace*{-0.5em}
     \end{minipage}\vspace*{-1.5em}
     
     \begin{minipage}[h!]{0.9\textwidth}
         \begin{subfigure}[b!]{0.3 \textwidth}
            \caption{$\epsilon = 5\%$}             \includegraphics[width=\textwidth]{Imagenes/Resultados_Hoti_Fractal/bands_square_shh_0.1.pdf}
         \end{subfigure}\hspace*{-0.5em}
         \begin{subfigure}[b!]{0.3 \textwidth}
            \caption*{}
            \includegraphics[width=\textwidth]{Imagenes/Resultados_Hoti_Fractal/bands_square_shh_log0.1.pdf}
         \end{subfigure}\hspace*{-0.5em}
         \begin{subfigure}[b!]{0.4 \textwidth}
            \caption*{}
            \includegraphics[width=\textwidth]{Imagenes/Resultados_Hoti_Fractal/proyection_square_0.1.pdf}
         \end{subfigure}\hspace*{-0.5em}
     \end{minipage}\vspace*{-1.5em}
     
     \begin{minipage}[h!]{0.9\textwidth}
         \begin{subfigure}[b!]{0.3 \textwidth}
            \caption{$\epsilon = 20\%$}             \includegraphics[width=\textwidth]{Imagenes/Resultados_Hoti_Fractal/bands_square_shh_0.2.pdf}
         \end{subfigure}\hspace*{-0.5em}
         \begin{subfigure}[b!]{0.3 \textwidth}
            \caption*{}
            \includegraphics[width=\textwidth]{Imagenes/Resultados_Hoti_Fractal/bands_square_shh_log0.2.pdf}
         \end{subfigure}\hspace*{-0.5em}
         \begin{subfigure}[b!]{0.4 \textwidth}
            \caption*{}
            \includegraphics[width=\textwidth]{Imagenes/Resultados_Hoti_Fractal/proyection_square_0.2.pdf}
         \end{subfigure}\hspace*{-0.5em}
     \end{minipage}\vspace*{-1.5em}
     
     \begin{minipage}[h!]{0.9\textwidth}
         \begin{subfigure}[b!]{0.3 \textwidth}
            \caption{$\epsilon = 30\%$}             \includegraphics[width=\textwidth]{Imagenes/Resultados_Hoti_Fractal/bands_square_shh_0.3.pdf}
         \end{subfigure}\hspace*{-0.5em}
         \begin{subfigure}[b!]{0.3 \textwidth}
            \caption*{}
            \includegraphics[width=\textwidth]{Imagenes/Resultados_Hoti_Fractal/bands_square_shh_log0.3.pdf}
         \end{subfigure}\hspace*{-0.5em}
         \begin{subfigure}[b!]{0.4 \textwidth}
            \caption*{}
            \includegraphics[width=\textwidth]{Imagenes/Resultados_Hoti_Fractal/proyection_square_0.3.pdf}
         \end{subfigure}\hspace*{-0.5em}
     \end{minipage}\vspace*{-1.5em}
     
      \begin{minipage}[h!]{0.9\textwidth}
         \begin{subfigure}[b!]{0.3 \textwidth}
            \caption{$\epsilon = 50\%$}             \includegraphics[width=\textwidth]{Imagenes/Resultados_Hoti_Fractal/bands_square_shh_0.5.pdf}
         \end{subfigure}\hspace*{-0.5em}
         \begin{subfigure}[b!]{0.3 \textwidth}
            \caption*{}
            \includegraphics[width=\textwidth]{Imagenes/Resultados_Hoti_Fractal/bands_square_shh_log0.5.pdf}
         \end{subfigure}\hspace*{-0.5em}
         \begin{subfigure}[b!]{0.4 \textwidth}
             \caption*{}
             \includegraphics[width=\textwidth]{Imagenes/Resultados_Hoti_Fractal/proyection_square_0.5.pdf}
         \end{subfigure}\hspace*{-0.5em}
     \end{minipage}\vspace*{-0.5em}
     
     
    \caption{En la columna derecha se muestra el espectro de energías en una red cuadrada como función de $\gamma/\lambda $ con un desorden aleatorio en los parámetros de salto, de tamaño $\epsilon$. Las lineas rojas corresponden a los cuatro estados degenerados que representan los estados localizados en las esquinas. En la columna central se muestra el espectro de energías en escala logarítmica. En la columna izquierda se muestra la densidad de probabilidad de la fase no trivial donde $\gamma' = \gamma( 1 \pm \epsilon) ,\, \, \lambda' = \lambda( 1 \pm \epsilon)$.  }
    \label{fig:para_proy_Delta_fractal}
\end{figure}

\subsection{Bombeo en red Cuadrada}
%% pump_cuadrado

En las figuras \ref{fig:Pump_cuadrado_x}, \ref{fig:Pump_cuadrado_y} y \ref{fig:Pump_cuadrado_xy} se muestran los resultados para el bombeo sobre una red cristalina cuadrada con variaciones cíclicas de los parámetros de salto en las direcciones $x$, $y$ y $xy$, respectivamente con un modelo de Rice-Mele aplicado a una red 2-dimensional. Notemos que la variación del espectro de energías tenemos dos clases de comportamiento conforme varia el parámetro cíclico $\theta$. La fase trivial donde el comportamiento donde las energías mas cercanas centrales mas cercanas al nivel de Fermi se comportan como parte del bulto, con $\theta \in [-\pi/2, \pi/2]$, y la fase topológica donde las energías centrales de borde comienzan a separarse de las energías de bulto en $\theta \in (-\pi,-\pi/2] \cup [\pi/2,\pi)$ hasta cerrarse completamente en $\theta = \pi,\pi$. El parámetro de sitio $\delta(\theta)$ nos permite separar la degeneración de la trayectoria de las energías de 4 a 2 energías por encima y 2 por debajo del nivel de Fermi, sin embargo, estas energías se vuelven a juntar en $\theta = \pi,\pi$ donde vuelven a ser indistinguibles, por lo que este estudio solo se hizo en el intervalo abierto $\theta \in (-\pi,\pi)$ . 
La proyección de la densidad de probabilidad de estados correspondientes a estas dos trayectorias de las energías positivas y negativas se muestran en la figura \ref{fig:Proy_pump_xy} [\textbf{(a)-(b)}], respectivamente, donde podemos ver como la densidad de probabilidad comienza localizada en 2 de las esquinas con $\rho_{corner} = 1/2$ y se comienza dispersar por todo el material hasta localizarse en las esquinas contrarias a las que comenzó, nuevamente con $\rho_{corner} = 1/2$, este proceso podría se un indicio de bombeo de densidad de probabilidad, pues estamos moviendo estas densidades por todo el material de un lado al otro, pasando por todo el bulto del material. 

Las flechas que aparecen en la figura \ref{fig:Proy_pump_xy} [\textbf{(a)-(b)}] son la densidad de corriente por cada celda \ref{eq:current_density_cell}:
\begin{equation}
    \label{eq:current_density_cell}
    \mathbf{J}^{\pm}(\theta) = \frac{1}{N_{cell}}\sum_{i \in cell} \frac{d \rho_{i,\pm}(\theta)}{d\theta} \mathbf{R}_i  
\end{equation}

Donde el signo $\pm$ corresponde a la trayectoria de energía que se elija, $\mathbf{R_i}$ es la posición del $i-$esimo sitio en la celda y $N_{cell}$ el numero de sitios en la celda. Esta corriente nos permite visualizar la dirección y magnitud del flujo de la densidad de probabilidad en cada celda. En las figuras \ref{fig:Pump_cuadrado_xy} [\textbf{(b, c)}] se visualiza el espacio fase que involucra, los parámetros de salto $\gamma/\lambda$, la densidad de corriente promedio por cuadrante \ref{eq:current_density_cuadrant} en el eje x, donde $N_C$ es el número de celdas en cada uno de los 4 cuadrantes de la red cristalina cuadrada \footnote{Para esta gráfica se tomo en cuenta el cuadrante inferior izquierdo}, y como estos evolucionan en el parámetro cíclico,  donde podemos ver que la corriente define una especie de curva que divide las dos fases de la red cristalina: al interior de la curva tendríamos la fase trivial y al exterior tendríamos la fase topológica, delimitadas por los puntos donde existe mayor corriente, Los puntos naranjas describen cuando el flujo de la densidad de probabilidad hacia dentro del cuadrante es mayor y los puntos azules describen cuando el flujo de la densidad hacia afuera del cuadrante es mayor. Los parámetros de $\gamma/\lambda \approx 1$ son para los cuales la corriente se ve mas favorecida y coinciden con los puntos en los cuales la transición de fase sucede en $\theta \approx \pi/2 $ o $-\pi/2$.

\begin{equation}
    \label{eq:current_density_cuadrant}
    \bar{J}^{\pm}_x(\theta) =\frac{1}{N_C N_{cell}} \sum_{C} \sum_{i \in cell} \frac{d \rho_{i,\pm}(\theta)}{d\theta} \mathbf{R}_i \cdot \mathbf{\hat{x}}  
\end{equation}

Sin embargo para poder asegurar que existe un bombeo consecuencia del cambio de la polarización generado por la variación de los parámetros de salto cíclicamente es necesario que exista un desplazamiento efectivo de los centros de Wannier \comA{Agregar ref ecuación} como se observa en el ejemplo del modelo de Rice-Mele en 1D en la cadena de poliacetileno. En la figura \ref{fig:Pump_cuadrado_xy} \texbf{(e)} muestra en negro el valor esperado de la posición de los estados con energías positivas para el modelo donde el bombeo sucede en las direcciones $xy$ donde podemos que existen centros de wannier los cuales comienzan en un una celda y después de un ciclo estos se desplazan a la siguiente celda o a la anterior, el punto de de desplazamiento de una celda a otra ocurre en el régimen de la fase topológica. En rojo podemos ver el valor esperado de la posición restringida a las energías mas cercanas a la energía de Fermi que fueran positivas, que al final son las energías que proyectamos en el bombeo, podemos ver su desplazamiento promedio por todo la red cristalina, como comienza en una esquina y después de un ciclo termina en otra.

El bombeo de densidad de probabilidad se da por este desplazamiento del valores esperado de la posición o centros de wannier de un sitio a otro. Para la red cristalina cuadrada fue posible encontrar parámetros de $\gamma/\lambda$ para los cuales pudiéramos ver este desplazamiento cuando el bombeo correspondiera a la variación de parámetros de salto en las direcciones $x$ y $y$, en las figuras \ref{fig:Pump_cuadrado_x} \texbf{(e)} y \ref{fig:Pump_cuadrado_x} \texbf{(e)} se puede ver claramente este comportamiento. Lo interesante es que estos bombeos se realizan moviendo las densidades principalmente por los bordes de la red cristalina \ref{fig:Proy_pump_x} y \ref{fig:Proy_pump_y} [\textbf{(a)-(b)}] y no principalmente por el bulto \ref{fig:Proy_pump_x}  [\textbf{(a)-(b)}], lo cual puede permitir burlar ``obstáculos'' como sucede en la red fractal.

\begin{figure}[h!]
     \centering
    \captionsetup[sub]{font=small}
     \begin{minipage}[h!]{1\textwidth}
         \begin{subfigure}[b!]{0.3 \textwidth}
             \caption{}
             \includegraphics[width=\textwidth]{Imagenes/Resultados_pump_Cuadrado/xy/param_pump_A=0.5.pdf}
         \end{subfigure}\hspace*{-0.5em}
         \begin{subfigure}[b!]{0.35 \textwidth}
             \caption{}
             \includegraphics[width=\textwidth]{Imagenes/Resultados_pump_Cuadrado/xy/current_square_pump.pdf}
         \end{subfigure}\hspace*{-0.5em}
         \begin{subfigure}[b!]{0.35 \textwidth}
             \caption{}
             \includegraphics[width=\textwidth]{Imagenes/Resultados_pump_Cuadrado/xy/current_square_pump_neg.pdf}
         \end{subfigure}\hspace*{-0.5em}
     \end{minipage}\vspace*{-1em}
     
     
     \begin{minipage}[h!]{1\textwidth}
        % \begin{subfigure}[b!]{0.37 \textwidth}
         %    \caption{}
          %   \includegraphics[width=\textwidth]{Imagenes/Resultados_pump_Cuadrado/xy/current_square_pump_pn.pdf}
           %  \label{}
         %\end{subfigure}\hspace{-0.5em}
         \begin{subfigure}[b!]{0.9 \textwidth}
             \caption{}
             \includegraphics[width=\textwidth]{Imagenes/Resultados_pump_Cuadrado/xy/wannier_centerx.pdf}
         \end{subfigure}\hspace*{-0.5em}
     \end{minipage}\vspace*{-1em}

     \begin{minipage}[h!]{1\textwidth}
        \begin{subfigure}[b!]{0.9 \textwidth}
            \caption{}
            \includegraphics[width=\textwidth]{Imagenes/Resultados_pump_Cuadrado/xy/wannier_centery.pdf}
        \end{subfigure}\hspace*{-0.5em}
    \end{minipage}\vspace*{-1em}
    
    \caption{\textbf{(a)} Variación del espectro de energías conforme cambia el parametro ciclico, las energias centrales corresponden a los estados de borde. En rojo se encuentra la fase topologica y en azul la fase trivial. \textbf{(a)}}
    \label{fig:Pump_cuadrado_xy}
\end{figure}


\begin{figure}[h!]
     \centering
    \captionsetup[sub]{font=small}
     \begin{minipage}[h!]{1.1\textwidth}
         \begin{subfigure}[b!]{0.2 \textwidth}
             \caption{$\theta = -\pi$}
             \includegraphics[width=\textwidth]{Imagenes/Resultados_pump_Cuadrado/xy/hoti_pomp_xy_pos1.pdf}
         \end{subfigure}\hspace*{-0.5em}
          \begin{subfigure}[b!]{0.2 \textwidth}
             \caption*{$\theta = -\frac{\pi}{2}$}
             \includegraphics[width=\textwidth]{Imagenes/Resultados_pump_Cuadrado/xy/hoti_pomp_xy_pos2.pdf}
         \end{subfigure}\hspace*{-0.5em}
          \begin{subfigure}[b!]{0.2 \textwidth}
             \caption*{$\theta = 0$}
             \includegraphics[width=\textwidth]{Imagenes/Resultados_pump_Cuadrado/xy/hoti_pomp_xy_pos3.pdf}
         \end{subfigure}\hspace*{-0.5em}
          \begin{subfigure}[b!]{0.2 \textwidth}
             \caption*{$\theta = \frac{\pi}{2}$}
             \includegraphics[width=\textwidth]{Imagenes/Resultados_pump_Cuadrado/xy/hoti_pomp_xy_pos4.pdf}
         \end{subfigure}\hspace*{-0.5em}
          \begin{subfigure}[b!]{0.2 \textwidth}
             \caption*{$\theta = \pi$}
             \includegraphics[width=\textwidth]{Imagenes/Resultados_pump_Cuadrado/xy/hoti_pomp_xy_pos5.pdf}
         \end{subfigure}\hspace*{-0.5em}
     \end{minipage}\vspace*{-1em}
     
     
     \begin{minipage}[h!]{1.1\textwidth}
          \begin{subfigure}[b!]{0.2 \textwidth}
             \caption{$\theta = -\pi$}
             \includegraphics[width=\textwidth]{Imagenes/Resultados_pump_Cuadrado/xy/hoti_pomp_xy_neg1.pdf}
         \end{subfigure}\hspace*{-0.5em}
          \begin{subfigure}[b!]{0.2 \textwidth}
             \caption*{$\theta = -\frac{\pi}{2}$}
             \includegraphics[width=\textwidth]{Imagenes/Resultados_pump_Cuadrado/xy/hoti_pomp_xy_neg2.pdf}
         \end{subfigure}\hspace*{-0.5em}
          \begin{subfigure}[b!]{0.2 \textwidth}
             \caption*{$\theta = 0$}
             \includegraphics[width=\textwidth]{Imagenes/Resultados_pump_Cuadrado/xy/hoti_pomp_xy_neg3.pdf}
         \end{subfigure}\hspace*{-0.5em}
          \begin{subfigure}[b!]{0.2 \textwidth}
             \caption*{$\theta = \frac{\pi}{2}$}
             \includegraphics[width=\textwidth]{Imagenes/Resultados_pump_Cuadrado/xy/hoti_pomp_xy_neg4.pdf}
         \end{subfigure}\hspace*{-0.5em}
          \begin{subfigure}[b!]{0.2 \textwidth}
             \caption*{$\theta = \pi$}
             \includegraphics[width=\textwidth]{Imagenes/Resultados_pump_Cuadrado/xy/hoti_pomp_xy_neg5.pdf}
         \end{subfigure}\hspace*{-0.5em}
     \end{minipage}\vspace*{-1em}
     
    \caption{}
    \label{fig:Proy_pump_xy}
\end{figure}


\begin{figure}[h!]
     \centering
    \captionsetup[sub]{font=small}
     \begin{minipage}[h!]{1\textwidth}
         \begin{subfigure}[b!]{0.3 \textwidth}
             \caption{}
             \includegraphics[width=\textwidth]{Imagenes/Resultados_pump_Cuadrado/x/param_pump_A=0.5x.pdf}
             \label{}
         \end{subfigure}\hspace*{-0.5em}
         \begin{subfigure}[b!]{0.35 \textwidth}
             \caption{}
             \includegraphics[width=\textwidth]{Imagenes/Resultados_pump_Cuadrado/x/current_square_pumpx.pdf}
             \label{}
         \end{subfigure}\hspace*{-0.5em}
         \begin{subfigure}[b!]{0.35 \textwidth}
             \caption{}
             \includegraphics[width=\textwidth]{Imagenes/Resultados_pump_Cuadrado/x/current_square_pump_negx.pdf}
             \label{}
         \end{subfigure}\hspace*{-0.5em}
     \end{minipage}\vspace*{-1em}
     
     
     \begin{minipage}[h!]{1\textwidth}
         \begin{subfigure}[b!]{0.37 \textwidth}
             \caption{}
             \includegraphics[width=\textwidth]{Imagenes/Resultados_pump_Cuadrado/x/current_square_pump_pnx.pdf}
             \label{}
         \end{subfigure}\hspace{-0.5em}
         \begin{subfigure}[b!]{0.63 \textwidth}
             \caption{}
             \includegraphics[width=\textwidth]{Imagenes/Resultados_pump_Cuadrado/x/wannier_centerx.pdf}
             \label{}
         \end{subfigure}\hspace*{-0.5em}
     \end{minipage}\vspace*{-1em}
     
     
    \caption{}
    \label{fig:Pump_cuadrado_x}
\end{figure}


\begin{figure}[h!]
     \centering
    \captionsetup[sub]{font=small}
     \begin{minipage}[h!]{1.1\textwidth}
         \begin{subfigure}[b!]{0.2 \textwidth}
             \caption{$\theta = -\pi$}
             \includegraphics[width=\textwidth]{Imagenes/Resultados_pump_Cuadrado/x/hoti_pomp_x_pos1.pdf}
             \label{}
         \end{subfigure}\hspace*{-0.5em}
          \begin{subfigure}[b!]{0.2 \textwidth}
             \caption*{$\theta = -\frac{\pi}{2}$}
             \includegraphics[width=\textwidth]{Imagenes/Resultados_pump_Cuadrado/x/hoti_pomp_x_pos2.pdf}
             \label{}
         \end{subfigure}\hspace*{-0.5em}
          \begin{subfigure}[b!]{0.2 \textwidth}
             \caption*{$\theta = 0$}
             \includegraphics[width=\textwidth]{Imagenes/Resultados_pump_Cuadrado/x/hoti_pomp_x_pos3.pdf}
             \label{}
         \end{subfigure}\hspace*{-0.5em}
          \begin{subfigure}[b!]{0.2 \textwidth}
             \caption*{$\theta = \frac{\pi}{2}$}
             \includegraphics[width=\textwidth]{Imagenes/Resultados_pump_Cuadrado/x/hoti_pomp_x_pos4.pdf}
             \label{}
         \end{subfigure}\hspace*{-0.5em}
          \begin{subfigure}[b!]{0.2 \textwidth}
             \caption*{$\theta = \pi$}
             \includegraphics[width=\textwidth]{Imagenes/Resultados_pump_Cuadrado/x/hoti_pomp_x_pos5.pdf}
             \label{}
         \end{subfigure}\hspace*{-0.5em}
     \end{minipage}\vspace*{-1em}
     
     
     \begin{minipage}[h!]{1.1\textwidth}
          \begin{subfigure}[b!]{0.2 \textwidth}
             \caption{$\theta = -\pi$}
             \includegraphics[width=\textwidth]{Imagenes/Resultados_pump_Cuadrado/x/hoti_pomp_x_neg1.pdf}
             \label{}
         \end{subfigure}\hspace*{-0.5em}
          \begin{subfigure}[b!]{0.2 \textwidth}
             \caption*{$\theta = -\frac{\pi}{2}$}
             \includegraphics[width=\textwidth]{Imagenes/Resultados_pump_Cuadrado/x/hoti_pomp_x_neg2.pdf}
             \label{}
         \end{subfigure}\hspace*{-0.5em}
          \begin{subfigure}[b!]{0.2 \textwidth}
             \caption*{$\theta = 0$}
             \includegraphics[width=\textwidth]{Imagenes/Resultados_pump_Cuadrado/x/hoti_pomp_x_neg3.pdf}
             \label{}
         \end{subfigure}\hspace*{-0.5em}
          \begin{subfigure}[b!]{0.2 \textwidth}
             \caption*{$\theta = \frac{\pi}{2}$}
             \includegraphics[width=\textwidth]{Imagenes/Resultados_pump_Cuadrado/x/hoti_pomp_x_neg4.pdf}
             \label{}
         \end{subfigure}\hspace*{-0.5em}
          \begin{subfigure}[b!]{0.2 \textwidth}
             \caption*{$\theta = \pi$}
             \includegraphics[width=\textwidth]{Imagenes/Resultados_pump_Cuadrado/x/hoti_pomp_x_neg5.pdf}
             \label{}
         \end{subfigure}\hspace*{-0.5em}
     \end{minipage}\vspace*{-1em}
     
    
     
     
    \caption{}
    \label{fig:Proy_pump_x}
\end{figure}


\begin{figure}[h!]
     \centering
    \captionsetup[sub]{font=small}
     \begin{minipage}[h!]{1\textwidth}
         \begin{subfigure}[b!]{0.3 \textwidth}
             \caption{}
             \includegraphics[width=\textwidth]{Imagenes/Resultados_pump_Cuadrado/y/param_pump_A=0.5y.pdf}
         \end{subfigure}\hspace*{-0.5em}
         \begin{subfigure}[b!]{0.35 \textwidth}
             \caption{}
             \includegraphics[width=\textwidth]{Imagenes/Resultados_pump_Cuadrado/y/current_square_pumpy.pdf}
         \end{subfigure}\hspace*{-0.5em}
         \begin{subfigure}[b!]{0.35 \textwidth}
             \caption{}
             \includegraphics[width=\textwidth]{Imagenes/Resultados_pump_Cuadrado/y/current_square_pump_negy.pdf}
         \end{subfigure}\hspace*{-0.5em}
     \end{minipage}\vspace*{-1em}
     
     
     \begin{minipage}[h!]{1\textwidth}
         \begin{subfigure}[b!]{1.0 \textwidth}
             \caption{}
             \includegraphics[width=\textwidth]{Imagenes/Resultados_pump_Cuadrado/y/wannier_centerx.pdf}
         \end{subfigure}\hspace*{-0.5em}
     \end{minipage}\vspace*{-1em}

     \begin{minipage}[h!]{1\textwidth}
        \begin{subfigure}[b!]{1.0 \textwidth}
            \caption{}
            \includegraphics[width=\textwidth]{Imagenes/Resultados_pump_Cuadrado/y/wannier_centery.pdf}
        \end{subfigure}\hspace*{-0.5em}
    \end{minipage}\vspace*{-0.5em}
     
     
    \caption{\textbf{(a)} Variación del espectro de energías conforme cambia el parametro ciclico, las energias centrales corresponden a los estados de borde. En rojo se encuentra la fase topologica y en azul la fase trivial. \textbf{(b)-(c)} Flujo de densidad de corriente para diferentes elecciones de $\gamma$ y $\lambda$ conforme cambia el parametro $\theta$, para el espectro positivo y el espectro negativo de energias de borde, respectivamente. \textbf{(d)-(e)} Cambio de los centros de Wannier para el espectro de energias negativas, en \textbf{X} y \textbf{Y}. Estos resultados corresponden al modelo de bombeo BBH en una red cuadrada en las direcciones \textbf{Y}.}
    \label{fig:Pump_cuadrado_y}
\end{figure}


\input{Images_Resultados/Pump_cuadrado_y_proy}




\subsection{Bombeo en red de Sierpinski}
%% Pump_fractal
-Comparacion del espectro de energias
- mencionar que en las gráficas de las energías la zona entre topológico y trivial es altamente deformada 
- sin embargo en las gráficas del bombeo es posible ver un desplazamiento de las cargas.
-Este desplazamiento de las cargas es corrobórale en las gráficas de los centros de wannier para la variación de parámetros en $x$ y $y$ sin embargo en la gráfica de para el bombeo en $XY$ esto no se corrobora pues el valor medio de la posición para la energías mas cercanas a la energía de fermi no logran desplazarse por toda la red fractal. 



\begin{figure}[h!]
     \centering
    \captionsetup[sub]{font=small}
     \begin{minipage}[h!]{1\textwidth}
         \begin{subfigure}[b!]{0.3 \textwidth}
             \caption{}
             \includegraphics[width=\textwidth]{Imagenes/Resultados_pump_Fractal/xy/param_pump_A=0.5.pdf}
             \label{}
         \end{subfigure}\hspace*{-0.5em}
         \begin{subfigure}[b!]{0.35 \textwidth}
             \caption{}
             \includegraphics[width=\textwidth]{Imagenes/Resultados_pump_Fractal/xy/current_square_pump.pdf}
             \label{}
         \end{subfigure}\hspace*{-0.5em}
         \begin{subfigure}[b!]{0.35 \textwidth}
             \caption{}
             \includegraphics[width=\textwidth]{Imagenes/Resultados_pump_Fractal/xy/current_square_pump_neg.pdf}
             \label{}
         \end{subfigure}\hspace*{-0.5em}
     \end{minipage}\vspace*{-1em}
     
     
     \begin{minipage}[h!]{1\textwidth}
         \begin{subfigure}[b!]{0.37 \textwidth}
             \caption{}
             \includegraphics[width=\textwidth]{Imagenes/Resultados_pump_Fractal/xy/current_square_pump_pn.pdf}
             \label{}
         \end{subfigure}\hspace{-0.5em}
         \begin{subfigure}[b!]{0.63 \textwidth}
             \caption{}
             \includegraphics[width=\textwidth]{Imagenes/Resultados_pump_Fractal/xy/wannier_center.pdf}
             \label{}
         \end{subfigure}\hspace*{-0.5em}
     \end{minipage}\vspace*{-1em}
     
     
    \caption{}
    \label{fig:Pump_fractal_xy}
\end{figure}


\begin{figure}[h!]
     \centering
    \captionsetup[sub]{font=small}
     \begin{minipage}[h!]{1.1\textwidth}
         \begin{subfigure}[b!]{0.2 \textwidth}
             \caption{$\theta = -\pi$}
             \includegraphics[width=\textwidth]{Imagenes/Resultados_pump_Fractal/xy/hoti_pomp_xy_pos1.pdf}
         \end{subfigure}\hspace*{-0.5em}
          \begin{subfigure}[b!]{0.2 \textwidth}
             \caption*{$\theta = -\frac{\pi}{2}$}
             \includegraphics[width=\textwidth]{Imagenes/Resultados_pump_Fractal/xy/hoti_pomp_xy_pos2.pdf}
         \end{subfigure}\hspace*{-0.5em}
          \begin{subfigure}[b!]{0.2 \textwidth}
             \caption*{$\theta = 0$}
             \includegraphics[width=\textwidth]{Imagenes/Resultados_pump_Fractal/xy/hoti_pomp_xy_pos3.pdf}
         \end{subfigure}\hspace*{-0.5em}
          \begin{subfigure}[b!]{0.2 \textwidth}
             \caption*{$\theta = \frac{\pi}{2}$}
             \includegraphics[width=\textwidth]{Imagenes/Resultados_pump_Fractal/xy/hoti_pomp_xy_pos4.pdf}
         \end{subfigure}\hspace*{-0.5em}
          \begin{subfigure}[b!]{0.2 \textwidth}
             \caption*{$\theta = \pi$}
             \includegraphics[width=\textwidth]{Imagenes/Resultados_pump_Fractal/xy/hoti_pomp_xy_pos5.pdf}
         \end{subfigure}\hspace*{-0.5em}
     \end{minipage}\vspace*{-1em}
     
     
     \begin{minipage}[h!]{1.1\textwidth}
          \begin{subfigure}[b!]{0.2 \textwidth}
             \caption{$\theta = -\pi$}
             \includegraphics[width=\textwidth]{Imagenes/Resultados_pump_Fractal/xy/hoti_pomp_xy_neg1.pdf}
         \end{subfigure}\hspace*{-0.5em}
          \begin{subfigure}[b!]{0.2 \textwidth}
             \caption*{$\theta = -\frac{\pi}{2}$}
             \includegraphics[width=\textwidth]{Imagenes/Resultados_pump_Fractal/xy/hoti_pomp_xy_neg2.pdf}
         \end{subfigure}\hspace*{-0.5em}
          \begin{subfigure}[b!]{0.2 \textwidth}
             \caption*{$\theta = 0$}
             \includegraphics[width=\textwidth]{Imagenes/Resultados_pump_Fractal/xy/hoti_pomp_xy_neg3.pdf}
         \end{subfigure}\hspace*{-0.5em}
          \begin{subfigure}[b!]{0.2 \textwidth}
             \caption*{$\theta = \frac{\pi}{2}$}
             \includegraphics[width=\textwidth]{Imagenes/Resultados_pump_Fractal/xy/hoti_pomp_xy_neg4.pdf}
         \end{subfigure}\hspace*{-0.5em}
          \begin{subfigure}[b!]{0.2 \textwidth}
             \caption*{$\theta = \pi$}
             \includegraphics[width=\textwidth]{Imagenes/Resultados_pump_Fractal/xy/hoti_pomp_xy_neg5.pdf}
         \end{subfigure}\hspace*{-0.5em}
     \end{minipage}\vspace*{-0.5em}
     
    
     
     
    \caption{\textbf{(a)-(b)} Proyeccion del cambio de la densidad de estados en la red de Sierpinski con bombeo BBH en las direcciones \textbf{XY}, las flechas corresponden al flujo de la densidad por celda $\mathbf{J}^{\pm}(\theta)$ [eq. \ref{eq:current_density_cell}]. Las figuras corresponden al especto positivo y negativo de las energias de borde, respectivamente}
    \label{fig:Proy_pump_fractal_xy}
\end{figure}


\begin{figure}[tbh!]
     \centering
    \captionsetup[sub]{font=small}
     \begin{minipage}[h!]{1\textwidth}
         \begin{subfigure}[b!]{0.3 \textwidth}
             \caption{}
             \includegraphics[width=\textwidth]{Imagenes/Resultados_pump_Fractal/x/param_pump_A=0.5x.pdf}
         \end{subfigure}\hspace*{-0.5em}
         \begin{subfigure}[b!]{0.35 \textwidth}
             \caption{}
             \includegraphics[width=\textwidth]{Imagenes/Resultados_pump_Fractal/x/current_square_pumpx.pdf}
         \end{subfigure}\hspace*{-0.5em}
         \begin{subfigure}[b!]{0.35 \textwidth}
             \caption{}
             \includegraphics[width=\textwidth]{Imagenes/Resultados_pump_Fractal/x/current_square_pump_negx.pdf}
         \end{subfigure}\hspace*{-0.5em}
     \end{minipage}\vspace*{-1em}
     
     
     \begin{minipage}[h!]{1\textwidth}
         \begin{subfigure}[b!]{1.0 \textwidth}
             \caption{}
             \includegraphics[width=\textwidth]{Imagenes/Resultados_pump_Fractal/x/wannier_centerx.pdf}
         \end{subfigure}\hspace*{-0.5em}
     \end{minipage}\vspace*{-1em}
     

     \begin{minipage}[h!]{1\textwidth}
        \begin{subfigure}[b!]{1.0 \textwidth}
            \caption{}
            \includegraphics[width=\textwidth]{Imagenes/Resultados_pump_Fractal/x/wannier_centery.pdf}
        \end{subfigure}\hspace*{-0.5em}
    \end{minipage}
     
    \caption{\textbf{(a)} Variación del espectro de energías conforme cambia el parametro ciclico, las energias centrales corresponden a los estados de borde. En rojo se encuentra la fase topologica y en azul la fase trivial. \textbf{(b)-(c)} Flujo de densidad de corriente para diferentes elecciones de $\gamma$ y $\lambda$ conforme cambia el parametro $\theta$, para el espectro positivo y el espectro negativo de energias de borde, respectivamente. \textbf{(d)-(e)} Cambio de los centros de Wannier para el espectro de energias negativas, en \textbf{X} y \textbf{Y}. Estos resultados corresponden al modelo de bombeo BBH en una red de Sierpinski en la dirección \textbf{X}.}
    \label{fig:Pump_fractal_x}
\end{figure}


\input{Images_Resultados/Pump_fractal_x_proy}


\begin{figure}[h!]
     \centering
    \captionsetup[sub]{font=small}
     \begin{minipage}[h!]{1\textwidth}
         \begin{subfigure}[b!]{0.3 \textwidth}
             \caption{}
             \includegraphics[width=\textwidth]{Imagenes/Resultados_pump_Fractal/y/param_pump_A=0.5y.pdf}
             \label{}
         \end{subfigure}\hspace*{-0.5em}
         \begin{subfigure}[b!]{0.35 \textwidth}
             \caption{}
             \includegraphics[width=\textwidth]{Imagenes/Resultados_pump_Fractal/y/current_square_pumpy.pdf}
             \label{}
         \end{subfigure}\hspace*{-0.5em}
         \begin{subfigure}[b!]{0.35 \textwidth}
             \caption{}
             \includegraphics[width=\textwidth]{Imagenes/Resultados_pump_Fractal/y/current_square_pump_negy.pdf}
             \label{}
         \end{subfigure}\hspace*{-0.5em}
     \end{minipage}\vspace*{-1em}
     
     
     \begin{minipage}[h!]{1\textwidth}
         \begin{subfigure}[b!]{0.37 \textwidth}
             \caption{}
             \includegraphics[width=\textwidth]{Imagenes/Resultados_pump_Fractal/y/current_square_pump_pny.pdf}
             \label{}
         \end{subfigure}\hspace{-0.5em}
         \begin{subfigure}[b!]{0.63 \textwidth}
             \caption{}
             \includegraphics[width=\textwidth]{Imagenes/Resultados_pump_Fractal/y/wannier_centery.pdf}
             \label{}
         \end{subfigure}\hspace*{-0.5em}
     \end{minipage}\vspace*{-1em}
     
     
    \caption{}
    \label{fig:Pump_fractal_y}
\end{figure}


\begin{figure}[h!]
     \centering
    \captionsetup[sub]{font=small}
     \begin{minipage}[h!]{1.1\textwidth}
         \begin{subfigure}[b!]{0.2 \textwidth}
             \caption{$\theta = -\pi$}
             \includegraphics[width=\textwidth]{Imagenes/Resultados_pump_Fractal/y/hoti_pomp_y_pos1.pdf}
             \label{}
         \end{subfigure}\hspace*{-0.5em}
          \begin{subfigure}[b!]{0.2 \textwidth}
             \caption*{$\theta = -\frac{\pi}{2}$}
             \includegraphics[width=\textwidth]{Imagenes/Resultados_pump_Fractal/y/hoti_pomp_y_pos2.pdf}
             \label{}
         \end{subfigure}\hspace*{-0.5em}
          \begin{subfigure}[b!]{0.2 \textwidth}
             \caption*{$\theta = 0$}
             \includegraphics[width=\textwidth]{Imagenes/Resultados_pump_Fractal/y/hoti_pomp_y_pos3.pdf}
             \label{}
         \end{subfigure}\hspace*{-0.5em}
          \begin{subfigure}[b!]{0.2 \textwidth}
             \caption*{$\theta = \frac{\pi}{2}$}
             \includegraphics[width=\textwidth]{Imagenes/Resultados_pump_Fractal/y/hoti_pomp_y_pos4.pdf}
             \label{}
         \end{subfigure}\hspace*{-0.5em}
          \begin{subfigure}[b!]{0.2 \textwidth}
             \caption*{$\theta = \pi$}
             \includegraphics[width=\textwidth]{Imagenes/Resultados_pump_Fractal/y/hoti_pomp_y_pos5.pdf}
             \label{}
         \end{subfigure}\hspace*{-0.5em}
     \end{minipage}\vspace*{-1em}
     
     
     \begin{minipage}[h!]{1.1\textwidth}
          \begin{subfigure}[b!]{0.2 \textwidth}
             \caption{$\theta = -\pi$}
             \includegraphics[width=\textwidth]{Imagenes/Resultados_pump_Fractal/y/hoti_pomp_y_neg1.pdf}
             \label{}
         \end{subfigure}\hspace*{-0.5em}
          \begin{subfigure}[b!]{0.2 \textwidth}
             \caption*{$\theta = -\frac{\pi}{2}$}
             \includegraphics[width=\textwidth]{Imagenes/Resultados_pump_Fractal/y/hoti_pomp_y_neg2.pdf}
             \label{}
         \end{subfigure}\hspace*{-0.5em}
          \begin{subfigure}[b!]{0.2 \textwidth}
             \caption*{$\theta = 0$}
             \includegraphics[width=\textwidth]{Imagenes/Resultados_pump_Fractal/y/hoti_pomp_y_neg3.pdf}
             \label{}
         \end{subfigure}\hspace*{-0.5em}
          \begin{subfigure}[b!]{0.2 \textwidth}
             \caption*{$\theta = \frac{\pi}{2}$}
             \includegraphics[width=\textwidth]{Imagenes/Resultados_pump_Fractal/y/hoti_pomp_y_neg4.pdf}
             \label{}
         \end{subfigure}\hspace*{-0.5em}
          \begin{subfigure}[b!]{0.2 \textwidth}
             \caption*{$\theta = \pi$}
             \includegraphics[width=\textwidth]{Imagenes/Resultados_pump_Fractal/y/hoti_pomp_y_neg5.pdf}
             \label{}
         \end{subfigure}\hspace*{-0.5em}
     \end{minipage}\vspace*{-1em}
     
    
     
     
    \caption{}
    \label{fig:Proy_pump_fractal_xy}
\end{figure}